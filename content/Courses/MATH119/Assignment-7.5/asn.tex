%%%%%%%%%%%%%%%%%%%%%%%%%%%%%%%%%%%%%%%%%% Preamble %%%%%%%%%%%%%%%%%%%%%%%%%%%%%%%%%%%%%%%%%%

\documentclass{assignment}
\usepackage[pdftex]{graphicx} % FIGURAS
\usepackage{xcolor}
\definecolor{LightGray}{gray}{0.95}
\usepackage[letterpaper, margin = 2.5cm]{geometry} % TAMAÑO DE PÁGINA Y MÁRGENES
\usepackage[T1]{fontenc} % Importante para acentos automáticos y símbolos de escritura
\usepackage{amsmath, amsfonts, amssymb} % Ecuaciones, caracteres y símbolos especiales
\usepackage{hyperref, url}
\usepackage{fancyhdr}

%-----------------------------------------ETIQUETAS--------------------------------------------

\student{Yashashwin Karthikeyan}
\semester{Winter 2024}
\date{\today}

\courselabel{MATH 119}
\exercisesheet{Tutorial 7.5}{Series}

\school{Dept. of ECE}
\university{University of Waterloo}

  %%%%%%%%%%%%%%%%%%%%%%%%%%%%%%%%%%%%%%%%%%-DOCUMENTO-%%%%%%%%%%%%%%%%%%%%%%%%%%%%%%%%%%%%%%%%%%%%

\begin{document}

  % Question 1
  \begin{problem}
  \section{}
    (a)

    \begin{flalign*}
      & \sum_{n = 1}^{\infty} \pi^{1 - 3n} &\\
      & = \sum_{n=1}^\infty \frac{\pi}{\pi^{3n}} &\\
      & = \sum_{n=1}^\infty \pi \Big(\frac{1}{\pi^{3}}\Big)^n &\\
    \end{flalign*}

    The above is a geometric series with common ratio $\frac{1}{\pi^3}$. Since $\Big|\frac{1}{\pi^3} \Big| \le 1$, the series converges.\\

    The first couple of terms are: $\{ \frac{1}{\pi^2}, \frac{1}{\pi^5}, \frac{1}{\pi^8} \} $.\\

    Re-indexing: $\sum_{n=0}^\infty \frac{1}{\pi^2} \Big(\frac{1}{\pi^3}\Big)^n $, where $a = \frac{1}{\pi^2}$ and $r = \frac{1}{\pi^3} $\\

    $\frac{a}{1 - r} = \frac{1/\pi^2}{1 - 1/\pi^3}$ converges to $\frac{\pi}{\pi^3 - 1}$, approximately 0.1047.\\

    $\implies \sum_{n=1}^{\infty} \pi^{1-3n}$ converges to approximately 0.1047.\\

    \noindent (b)
    \begin{flalign*}
      \sum_{n=1}^{\infty} \frac{3 + 4^n}{5^n}
    \end{flalign*}
    Applying ratio test:
    \begin{flalign*}
      \lim_{n\to\infty} \Big| \frac{a_{n+1}}{a_n} \Big| = \lim_{n\to\infty} \Big| \Big( \frac{3 + 4^{n+1}}{5^{n+1}} \Big ) \Big( \frac{5^n}{3 + 4^n} \Big ) \Big| = \lim_{n\to\infty} \Big | \frac{3 + 4^{n+1}}{5(3 + 4^n)} \Big | = \lim_{n\to\infty} \Big | \frac{3/4^n + 4}{5(3/4^n + 1)} \Big | = \frac 4 5\\
    \end{flalign*}

    \begin{flalign*}
      \lim_{n\to\infty} \Big| \frac{a_{n+1}}{a_n} \Big| \le 1\\
    \end{flalign*}

    Hence, the series is absolutely convergent.
  \end{problem}


  % Question 2
  \begin{problem}
  \section{More proofs about $p$ and $q$}
    \begin{flalign*}
      & \text{Type } Ty := AA || BB &\\
      & p, q: Ty \to \mathbb{B} \\\\
      & \vdash (\forall x: Ty . p(x) \to \neg q(x)) \to \\ 
      & \indent (\exists x: Ty . p(x)) \to\\
      & \indent p(AA) \to\\
      & \indent\exists y: Ty . q(y)\\\\
      & \text{By impl-elims on goal:}\\
      & \indent 1)\ \forall x: Ty . p(x) \to \neg q(x)\\
      & \indent 2)\ \exists x: Ty . \neg p(x)\\
      & \indent 3)\ p(\text{AA})\\\\
      & \vdash \exists y: Ty . q(y)\\
    \end{flalign*}
    \begin{tabular}{ |c|c|c| } 
      \hline
      $Ty$ & $p$ & $q$ \\\hline
      $AA$ & T & F \\\hline
      $BB$ & F & T/F \\\hline
    \end{tabular}\\\\

    The implication allows $q(BB)$ to be false while allowing all assumptions to be true. Hence, by
    counterexample the goal is false.
  \end{problem}

  % Question 3

  \begin{problem}
  \section{Proofs are contrary to fun}
    \begin{flalign*}
      & \text{Type } Ty &\\
      & foo: Ty \\
      & p, q: Ty \to \mathbb{B}\\\\
      & \vdash (\forall w: Ty.p(w) \implies \forall x: Ty .\neg q(x)) \implies\\
      & \indent (\exists y: Ty. q(y)) \implies\\
      & \indent p(foo) \implies\\
      & \indent (\forall z: Ty. q(z))\\\\
      & \text{By impl-elims on goal:}\\
      & \indent 1)\ \forall w: Ty\ (p(w) \implies \forall x: Ty . \neg q(x))\\
      & \indent 2)\ \exists y: Ty . q(y)\\
      & \indent 3)\ p(foo)\\\\
      & \vdash \forall z: Ty. q(z)\\\\
      & \text{By forall-elim on 1 using } w = foo\\
      & \indent 4)\ p(foo) \implies \forall x: Ty. \neg q(x) \\
      & \text{By impl-elim on 4 using 3}\\
      & \indent 5)\ \forall x: Ty. \neg q(x) \\
      & \text{By exists-elim on 2:}\\
      & \indent 6)\ y: Ty\\
      & \indent 7)\ q(y)\\
      & \text{By forall-elim on 5 using } x = y\\
      & \indent 8)\ \neg q(y) \\\\
      & \text{Assumption 5 contradicts 8, therefore the goal is false.}
    \end{flalign*}
  \end{problem}

  % Question 4

  \begin{problem}
  \section{Simple proofs can be sick}
    Type person, location, liquid\\
    visited: (person, location) $\to \mathbb{B}$ \\
    sick: person $\to \mathbb{B}$\\
    $ooj$: liquid (old orange juice)\\
    beach: location\\
    drank: (person, liquid) $\to \mathbb{B}$\\
    Marat: person \\\\

    \begin{enumerate}
      \item Everyone who drank old orange juice got sick.\\
        $\forall p: $ person . drank($p, ooj$) $\implies$ sick($p$)
      \item Everyone who drank old orange juice went to the beach.\\
        $\forall p: $ person . drank($p, ooj$) $\implies$ visited($p$, beach)
      \item Marat did not get sick.\\
        $\neg$ sick(Marat)
      \item To prove that the three statements imply that Marat did not go to the beach:\\\\
        \begin{tabular}{ |c|c|c|c| } 
          \hline
          $p$ & sick(p) & visited($p$, beach) & drank(p, $ooj$) \\\hline
          Marat & F & T & F \\\hline
        \end{tabular}\\\\

        The above environment illustrates a case where all assumptions hold, yet the goal is not
        satisfied. Therefore, by counterexample, the statements are insufficient to prove the goal.

    \end{enumerate}
  \end{problem}

\end{document}
